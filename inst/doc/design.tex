
\documentclass{article}

\usepackage{indentfirst}
\usepackage{apalike}
\usepackage{amsmath}
\usepackage{url}
\newcommand{\set}[1]{\{\, #1 \,\}}
\DeclareMathOperator{\pr}{Pr}
\DeclareMathOperator{\var}{var}
\DeclareMathOperator{\logit}{logit}
\newcommand{\BinomialDis}{\text{Bin}}
\newcommand{\NegativeBinomialDis}{\text{NegBin}}
\newcommand{\PoissonDis}{\text{Pois}}

\RequirePackage{amsthm}
\newtheorem{theorem}{Theorem}

\begin{document}

\title{Design of the \texttt{ump} Package}
\author{Charles J. Geyer \and Glen D. Meeden}
\maketitle

\section{Introduction}

\subsection{The UMP and UMPU Problems}

The following is mostly taken from the first draft of the fuzzy confidence
intervals and $P$-values paper.

\subsubsection{UMP}

Lehmann (1959, pp.~68--69) says for a one-parameter
model with likelihood ratio monotone in the statistic $T(X)$
there exists a UMP test
having null hypothesis $H_0 = \set{ \vartheta : \vartheta \le \theta }$,
alternative hypothesis $H_1 = \set{ \vartheta : \vartheta > \theta }$,
significance level $\alpha$,
and critical function $\phi$ defined by
\begin{equation} \label{eq:ump-crit}
   \phi(x, \alpha, \theta)
   =
   \begin{cases}
   1, & T(x) > C
   \\
   \gamma, & T(x) = C
   \\
   0, & T(x) < C
   \end{cases}
\end{equation}
where the constants $\gamma$ and $C$ are determined by
$$
   E_\theta\{\phi(X, \alpha, \theta)\} = \alpha.
$$
The description of the analogous lower-tailed test is
the same except that all inequalities are reversed.

The constant $C$ is clearly any $(1 - \alpha)$-th quantile of the distribution
of $T(X)$ for the parameter value $\theta$.
If $C$ is not an atom of this distribution, then
the test is effectively not randomized and the value of $\gamma$ is
irrelevant.  Otherwise
\begin{equation} \label{eq:gamma-ump}
   \gamma = \frac{\alpha - \pr_\theta\{T(X) > C\}}{\pr_\theta\{T(X) = C\}}.
\end{equation}

\subsubsection{UMPU} \label{sec:umpu}

Lehmann (1959, pp.~126--127) says for a one-parameter
exponential family model with canonical statistic $T(X)$ 
and canonical parameter $\theta$
there exists a UMPU test
having null hypothesis $H_0 = \set{ \vartheta : \vartheta = \theta }$,
alternative hypothesis $H_1 = \set{ \vartheta : \vartheta \neq \theta }$,
significance level $\alpha$,
and critical function $\phi$ defined by
\begin{equation} \label{eq:umpu-crit}
   \phi(x, \alpha, \theta)
   =
   \begin{cases}
   1, & T(x) < C_1
   \\
   \gamma_1, & T(x) = C_1
   \\
   0, & C_1 < T(x) < C_2
   \\
   \gamma_2, & T(x) = C_2
   \\
   1, & C_2 < T(x)
   \end{cases}
\end{equation}
where $C_1 \le C_2$ and the constants $\gamma_1$, $\gamma_2$, $C_1$, and $C_2$
are determined by
\begin{subequations}
\begin{align}
   E_\theta\{\phi(X, \alpha, \theta)\} & = \alpha
   \label{eq:umpu-a}
   \\
   E_\theta\{T(X) \phi(X, \alpha, \theta)\} & = \alpha E_\theta\{T(X)\}
   \label{eq:umpu-b}
\end{align}
\end{subequations}

If $C_1 = C_2 = C$ in \eqref{eq:umpu-crit} then $\gamma_1 = \gamma_2 = \gamma$
also.  This occurs only in a very special case.
Define
\begin{subequations}
\begin{align}
   p & = \pr_\theta\{T(X) = C\}
   \label{eq:p}
   \\
   \mu & = E_\theta\{T(X)\}
   \label{eq:mu}
\end{align}
\end{subequations}
Then in order to satisfy \eqref{eq:umpu-a} and \eqref{eq:umpu-b} we must have
\begin{align*}
   1 - (1 - \gamma) p & = \alpha
   \\
   \mu - C (1 - \gamma) p & = \alpha \mu
\end{align*}
which solved for $\gamma$ and $C$ gives
\begin{subequations}
\begin{align}
   \gamma & = 1 - \frac{1 - \alpha}{p}
   \label{eq:umpu-spec-a}
   \\
   C & = \mu
   \label{eq:umpu-spec-b}
\end{align}
\end{subequations}
Thus this special case occurs only when $\mu$
an atom of the distribution of $T(X)$ for the parameter value $\theta$,
and then only for
very large significance levels: $\alpha > 1 - p$.
Hence this special case is of no practical importance,
although
it is of some computational importance to get every case right, no weird
bogus results or crashes in unusual special cases.

Returning to the general case, assume for a second that we have
particular $C_1$ and $C_2$ that work for some $x$, $\alpha$, and $\theta$
(we will see how to determine $C_1$ and $C_2$ presently).
With $\mu$ still defined by \eqref{eq:mu} and with the definitions
\begin{subequations}
\begin{align}
   p_i & = \pr_\theta\{ T(X) = C_i \}, \qquad i = 1, 2
   \label{eq:pi}
   \\
   p_{1 2} & = \pr_\theta\{ C_1 < T(X) < C_2 \}
   \label{eq:p12}
   \\
   m_{1 2} & = E_\theta\{ T(X) I_{(C_1, C_2)}[T(X)] \}
   \label{eq:m12}
\end{align}
\end{subequations}
\eqref{eq:umpu-a} and \eqref{eq:umpu-b} become
\begin{subequations}
\begin{align}
   1 - (1 - \gamma_1) p_1 - (1 - \gamma_2) p_2 - p_{1 2}
   & = \alpha
   \label{eq:umpu-a-general}
   \\
   \mu - C_1 (1 - \gamma_1) p_1 - C_2 (1 - \gamma_2) p_2 - m_{1 2}
   & = \alpha \mu
   \label{eq:umpu-b-general}
\end{align}
\end{subequations}
which solved for $\gamma_1$ and $\gamma_2$ give
\begin{subequations}
\begin{align}
   \gamma_1
   & =
   1 - \frac{(1 - \alpha) (C_2 - \mu) + m_{1 2} - C_2 p_{1 2}}{p_1 (C_2 - C_1)}
   \label{eq:gamma-1}
   \\
   \gamma_2
   & =
   1 - \frac{(1 - \alpha) (\mu - C_1) - m_{1 2} + C_1 p_{1 2}}{p_2 (C_2 - C_1)}
   \label{eq:gamma-2}
\end{align}
\end{subequations}

Note that \eqref{eq:gamma-1} and \eqref{eq:gamma-2} are linear in $\alpha$.
They are valid over the range of $\alpha$ (if any) such that both equations
give values between zero and one.

Now we turn to the determination of $C_1$ and $C_2$.  We present an algorithm
\label{pg:algo}
that determines $\phi(x, \alpha, \theta)$ for any discrete one-parameter
exponential family with canonical statistic $T(X)$ for all values
of $x$ and $\alpha$ for one fixed value of $\theta$.
\begin{enumerate}
\item Start with $\alpha = 1$.
\begin{enumerate}
\item If $\mu$ given by \eqref{eq:mu} is an atom, then
$\phi(x, \alpha, \theta)$ is given by
\eqref{eq:umpu-crit} with $C_1 = C_2 = \mu$
and $\gamma_1 = \gamma_2 = \gamma$
given by \eqref{eq:p}, \eqref{eq:umpu-spec-a}, and \eqref{eq:umpu-spec-b}
over the range of $\alpha$ such that
\eqref{eq:umpu-spec-a} is between zero and one.
\item If $\mu$ given by \eqref{eq:mu} is not an atom, then choose $C_1$
and $C_2$ to be adjacent atoms such that $C_1 < \mu < C_2$ and
$\phi(x, \alpha, \theta)$ is given by \eqref{eq:umpu-crit}
with $\gamma_1$ and $\gamma_2$ given by
\eqref{eq:pi}, \eqref{eq:p12}, \eqref{eq:m12},
\eqref{eq:gamma-1}, and \eqref{eq:gamma-2}
over the range of $\alpha$ such that both
\eqref{eq:gamma-1} and \eqref{eq:gamma-2} are between zero and one.
[Because $C_1$ and $C_2$ are adjacent atoms, $p_{1 2} = m_{1 2} = 0$,
and $\alpha = 1$ gives $\gamma_1 = \gamma_2 = 1$, a valid solution.]
\end{enumerate}
\item Start with the smallest $\alpha$ for which $\phi(x, \alpha, \theta)$
was determined in step 1 or a previous iteration of step 2.
At this point, either $\gamma_1$ or $\gamma_2$
is zero (or both are).
\begin{enumerate}
\item If $\gamma_1$ is zero, then decrease $C_1$ to the adjacent lower
atom and set $\gamma_1 = 1$ [which does not change the value
of $\phi(x, \alpha, \theta)$ for any $x$].
\item If $\gamma_2$ is zero, then increase $C_2$ to the adjacent higher
atom and set $\gamma_2 = 1$ [which does not change the value
of $\phi(x, \alpha, \theta)$ for any $x$].
\item Now $\phi(x, \alpha, \theta)$ is given by \eqref{eq:umpu-spec-a}
with $\gamma_1$ and $\gamma_2$ given by
\eqref{eq:pi}, \eqref{eq:p12}, \eqref{eq:m12},
\eqref{eq:gamma-1}, and \eqref{eq:gamma-2}
over the range of $\alpha$ such that both
\eqref{eq:gamma-1} and \eqref{eq:gamma-2} are between zero and one
[because of steps (a) and (b), both $\gamma_i$ are now greater than
zero, so $\alpha$ can be decreased].
\end{enumerate}
\item Repeat step 2 until the whole range $0 \le \alpha \le 1$ is covered.
\end{enumerate}

This algorithm is certainly unwieldy, but it does make clear that
$\alpha \mapsto \phi(x, \alpha, \theta)$ is (1) continuous, (2) piecewise
linear, (3) nondecreasing, and (4) onto $[0, 1]$.
Hence it is the distribution function of a continuous
random variable (an abstract randomized $P$-value).
Clearly, it is differentiable on
each linear piece and the derivative is piecewise constant (a step function).

\subsection{Endpoint Behavior} \label{sec:endpoint}

The UMPU test is not well defined when the null hypothesis is on
the boundary of the parameter space.  But equations \eqref{eq:umpu-crit},
\eqref{eq:umpu-a},
and \eqref{eq:umpu-b} still make sense and define a test.  Since
the probability and the expectation in those equations are continuous
in $\theta$ this also characterizes the behavior as $\theta$ converges
to a boundary point (which we need to know to calculate fuzzy confidence
intervals, which involve all $\theta$ in the parameter space).

\begin{theorem} \label{th:umpu-end}
Suppose the setup for a UMPU test described
in the first paragraph of Section~\ref{sec:umpu}.
In addition suppose that $T(X)$ has a topologically discrete distribution
(concentrated on a countable set of atoms
and the atoms topologically isolated) not concentrated at one point.
If the support $S$ of $T(X)$ has a lower bound and $L$ is the two-point
set consisting of the two smallest atoms of the support, then
\begin{equation} \label{eq:umpu-end}
   1 - \phi(x, \alpha, \theta) \to (1 - \alpha) I_L(x),
   \qquad \text{as $\theta \to - \infty$}.
\end{equation}
Similarly, if the support has an upper bound and $L$ consists of the two
largest atoms, then \eqref{eq:umpu-end} holds with $- \infty$ replaced by
$+ \infty$.
\end{theorem}
\begin{proof}
We do the case where the support has a lower bound.  The upper bound
case is entirely analogous.

The densities in the family of distributions of $T(X)$ have the form
\begin{equation} \label{eq:expo-dens}
   f_\theta(s) = \frac{1}{c(\theta)} e^{s \theta} \lambda(s),
   \qquad s \in S,
\end{equation}
where $\lambda$ is some strictly positive function, and
\begin{equation} \label{eq:expo-norm}
   c(\theta) = \sum_{s \in S} e^{s \theta} \lambda(s).
\end{equation}

The natural parameter space of the family is the set $\Theta$ of $\theta$
such that \eqref{eq:expo-norm} is finite.  Because $S$ is bounded below,
if $c(\psi) < \infty$,
then $c(\theta) < \infty$ for all $\theta < \psi$.  Thus $\Theta$ is either
the whole real line or a semi-infinite interval extending to $- \infty$.

For every $\theta$ in the interior of $\Theta$, the distribution with
density $f_\theta$ has a moment generating function $M_\theta$ defined by
$$
   M_\theta(t) = E_\theta\{e^{t T(X)}\} = \frac{c(\theta + t)}{c(\theta)},
$$
and hence this distribution has moments of all orders, the mean and variance
being given by derivatives at zero of the cumulant generating function
$\log M_\theta$
\begin{align*}
   \mu(\theta) & = E_\theta\{T(X)\} = \frac{d}{d \theta} \log c(\theta)
   \\
   \sigma^2(\theta) & = \var_\theta\{T(X)\}
   = \frac{d^2}{d \theta^2} \log c(\theta)
\end{align*}
Because of our assumption that $T(X)$ is not concentrated at one point,
$\sigma^2(\theta) = d \mu(\theta) / d \theta$ can
never be zero.  Hence $\mu$ is
a strictly increasing continuous function that maps the interior of $\Theta$
to some open interval of the real line.

Write $L = \{ s_0, s_1 \}$ with $s_0 < s_1$.
Since
$$
   \frac{f_\theta(s)}{f_\theta(s_0)}
   = e^{(s - s_0) \theta} \frac{\lambda(s)}{\lambda(s_0)}
$$
the distribution clearly converges to the distribution concentrated at
$s_0$ as $\theta \to - \infty$.

Now
$$
   \frac{\mu(\theta) - s_0}{f_\theta(s_1)}
   =
   \sum_{s \in S \setminus \{s_0\}} (s - s_0) e^{(s - s_1) \theta}
   \frac{\lambda(s)}{\lambda(s_1)}
$$
goes to $s_1 - s_0$ by monotone convergence as $\theta \to - \infty$.
Hence $\mu(\theta) \to s_0$ as $\theta \to - \infty$.
Thus $\mu$ is a diffeomorphism from the interior of $\Theta$ to some
open interval of the real line, the lower endpoint of which is $s_0$.

Now
$$
   \frac{\pr_\theta\{T(X) > s_1\}}{f_\theta(s_1)}
   =
   \sum_{s \in S \setminus L}
   e^{(s - s_1) \theta} \frac{\lambda(s)}{\lambda(s_1)}
$$
goes to zero by monotone convergence as $\theta \to - \infty$.

And these facts together imply
\begin{equation} \label{eq:expo-gen}
\begin{split}
   \pr_\mu\{T(X) = s_0\} & = 1 - \frac{\mu - s_0}{s_1 - s_0} + o(\mu - s_0)
   \\
   \pr_\mu\{T(X) = s_1\} & = \frac{\mu - s_0}{s_1 - s_0} + o(\mu - s_0)
   \\
   \pr_\mu\{T(X) > s_1\} & = o(\mu - s_0)
\end{split}
\end{equation}
where $\mu = \mu(\theta)$ is the mean value parameter.

Now we claim that for small enough values of $\theta$ or $\mu$
we have $C_1 = s_0$ and $C_2 = s_1$ and the UMPU test is given by
equations (3.10a) and (3.10b) in the paper with $p_1$ and $p_2$
given by (3.8a) in the paper and $p_{1 2} = m_{1 2} = 0$.  Let's check.
These equations give now
\begin{align*}
   \gamma_1
   & =
   1 - \frac{(1 - \alpha) (s_1 - \mu)}{s_1 - \mu + o(\mu - s_0)}
   \\
   \gamma_2
   & =
   1 - \frac{(1 - \alpha) (\mu - s_0)}{\mu - s_0 + o(\mu - s_0)}
\end{align*}
and clearly both converge to $\alpha$ as $\mu \to s_0$ hence both are
between zero and one for small enough $\theta$ or $\mu$ and hence
define the UMPU test.
\end{proof}

This explains the behavior of the
fuzzy confidence intervals for the binomial distribution for
the two $x$ values nearest each boundary in Figure~2 of the paper.
As $\theta \to 0$, the fuzzy confidence interval $1 - \phi(x, \alpha, \theta)$
converges to $1 - \alpha$ for $x = 0$ or $x = 1$ and converges to zero
for all other $x$.
And as $\theta \to 1$, the fuzzy confidence interval
converges to $1 - \alpha$ for $x = n - 1$ or $x = n$ and converges to zero
for all other $x$.

\subsection{Models With Nuisance Parameters}

UMP and UMPU theory extends to multiparameter exponential families
when the parameter of interest $\theta$ is one of the canonical parameters
(Lehmann, TSH, 1st ed.,\ pp.\ 134--136).

Suppose the family has densities of the form
$$
   \frac{1}{c(\theta, \boldsymbol{\eta})}
   \exp\left( \theta T(x) + \sum_{i = 1}^k \eta_i U_i(x) \right)
$$
with respect to some measure on the sample space.  Then the situation
is exactly the same as described above except that the reference distribution
of the test is the conditional distribution of $T(X)$ given $\mathbf{U}(X)$,
which (a standard fact about exponential families) depends only on $\theta$
and not on the nuisance parameter $\boldsymbol{\eta}$.

\subsubsection{UMP Tests With Nuisance Parameters}

Now there exists a UMP test
having null hypothesis $H_0 = \set{ \vartheta : \vartheta \le \theta }$,
alternative hypothesis $H_1 = \set{ \vartheta : \vartheta > \theta }$,
and significance level $\alpha$,
and its critical function $\phi$ is defined by
\begin{equation} \label{eq:crit-lower}
\begin{split}
   \phi(x, \alpha, \theta)
   =
   \begin{cases}
   1, & T(x) > C[\mathbf{U}(x)]
   \\
   \gamma[\mathbf{U}(x)], & T(x) = C[\mathbf{U}(x)]
   \\
   0, & T(x) < C[\mathbf{U}(x)]
   \end{cases}
\end{split}
\end{equation}
where the functions $\gamma$ and $C$ are determined by
\begin{equation} \label{eq:cond-lower}
   E_\theta\{\phi(X, \alpha, \theta) \mid \mathbf{U}(X) \} = \alpha.
\end{equation}

Everything is exactly the same as for the one-parameter case except
for the conditioning on $\mathbf{U}(x)$.  The only point of the discussion
is that the test is UMP whether considered conditionally or unconditionally.

As before, the UMP upper-tailed test is obtained by reversing all the
inequalities above.

\subsubsection{UMPU Tests With Nuisance Parameters}

Now there exists a UMPU test
having null hypothesis $H_0 = \set{ \vartheta : \vartheta = \theta }$,
alternative hypothesis $H_1 = \set{ \vartheta : \vartheta \neq \theta }$,
and significance level $\alpha$,
and its critical function $\phi$ is defined by
\begin{equation} \label{eq:crit-two}
\begin{split}
   \phi(x, \alpha, \theta)
   =
   \begin{cases}
   1, & T(x) < C_1[\mathbf{U}(x)]
   \\
   \gamma_1[\mathbf{U}(x)], & T(x) = C_1[\mathbf{U}(x)]
   \\
   0, & C_1[\mathbf{U}(x)] < T(x) < C_2[\mathbf{U}(x)]
   \\
   \gamma_2[\mathbf{U}(x)], & T(x) = C_2[\mathbf{U}(x)]
   \\
   1, & C_2[\mathbf{U}(x)] < T(x)
   \end{cases}
\end{split}
\end{equation}
where the functions $\gamma_1$, $\gamma_2$, $C_1$, and $C_2$ are determined by
\begin{subequations}
\begin{align}
   E_\theta\{\phi(X, \alpha, \theta) \mid \mathbf{U}(X) \} & = \alpha
   \label{eq:cond-two-a}
   \\
   E_\theta\{T(X) \phi(X, \alpha, \theta) \mid \mathbf{U}(X) \} & = \alpha
   E_\theta\{T(X) \mid \mathbf{U}(X) \}
   \label{eq:cond-two-b}
\end{align}
\end{subequations}

Again, the point
is that the test is UMPU whether considered conditionally or unconditionally.

% As we saw in Section~\ref{sec:special}, it is possible that $C_1 = C_2$
% in which case $\gamma_1$ and $\gamma_2$ are not separately determinable,
% only their sum.  Analysis of that situation here would just repeat that
% section word for word except for references to conditioning on $\mathbf{U}(x)$.

\section{Models}

\subsection{Binomial}

Let $X \sim \BinomialDis(n, p)$ with $0 < p < 1$.

All of the quantities in \eqref{eq:gamma-1} and \eqref{eq:gamma-2}
are easily calculated (in R) except possibly $m_{1 2}$.  Actually,
as Lehmann points out (TSH, 1st, ed.,\ pp.~128--129), this quantity
is also easy to calculate
$$
   \sum_{x = C_1 + 1}^{C_2 - 1} x \binom{n}{x} p^x (1 - p)^{n - x}
   =
   n p \sum_{x = C_1 + 1}^{C_2 - 1} \binom{n - 1}{x - 1}
   p^{x - 1} (1 - p)^{(n - 1) - (x - 1)}
$$
and the sum on the right is just a binomial probability for the
$\BinomialDis(n - 1, p)$ distribution, that is, the right side can be calculated
in R (with the obvious definitions of the variables) by
\begin{verbatim}
n * p * (pbinom(c2 - 2, n - 1, p) - pbinom(c1 - 1, n - 1, p))
\end{verbatim}
assuming \texttt{pbinom(c1 - 1, n - 1, p)} does not crash and
produces zero when \texttt{c1} is zero (which it
does in recent versions I can check).

\subsection{Poisson}

Let $X \sim \PoissonDis(\mu)$ with $0 < \mu$.

Again, all of the quantities in \eqref{eq:gamma-1} and \eqref{eq:gamma-2}
are easily calculated except possibly $m_{1 2}$.  Does it work like the
binomial case?  Yes!
$$
   \sum_{x = C_1 + 1}^{C_2 - 1} x \frac{\mu^x}{x !} e^{- \mu}
   =
   \mu \sum_{x = C_1 + 1}^{C_2 - 1} \frac{\mu^{x - 1}}{(x - 1)!} e^{- \mu}
$$
and the sum on the right is just another Poisson probability,
that is, the right side can be calculated
in R (with the obvious definitions of the variables) by
\begin{verbatim}
mu * (ppois(c2 - 2, mu) - ppois(c1 - 1, mu))
\end{verbatim}

\subsection{Negative Binomial} \label{sec:negbin}

Let $X \sim \NegativeBinomialDis(r, p)$ with $0 < p < 1$.  Like R we consider
the sample space to start at zero rather than $r$.  This also allows for
non-integer $r$.  The densities of the family have the form
$$
   f(x) = \frac{\Gamma(x+r)}{\Gamma(r) x!} p^r (1-p)^x
$$

Note that if we are to have an exponential family $r$ cannot be an unknown
parameter!  The only unknown parameter is $p$.

Again, all of the quantities in \eqref{eq:gamma-1} and \eqref{eq:gamma-2}
are easily calculated except possibly $m_{1 2}$.  Does it work like the
binomial case?  Yes!
$$
   \sum_{x = C_1 + 1}^{C_2 - 1} x \frac{\Gamma(x+r)}{\Gamma(r) x!} p^r (1-p)^x
   =
   \frac{1 - p}{p} \sum_{x = C_1 + 1}^{C_2 - 1}
   \frac{\Gamma(x - 1 + r + 1)}{\Gamma(r) (x - 1)!} p^{r + 1} (1-p)^{x - 1}
$$
and the sum on the right is just another negative binomial probability,
that is, the right side can be calculated
in R (with the obvious definitions of the variables) by
\begin{verbatim}
(1 - p) / p * (pnbinom(c2 - 2, r + 1, p)
    - pnbinom(c1 - 1, r + 1, p))
\end{verbatim}

\subsection{Two Independent Poisson Random Variables}

Let $X_i \sim \PoissonDis(\mu_i)$ with $0 < \mu_i$, for $i = 1$, $2$
be independent random variables.  We wish to compare the means $\mu_1$ and
$\mu_2$.  We cannot just test or produce fuzzy confidence intervals for
a function pulled out of the air, such as $\mu_1 - \mu_2$.  The parameter
we test must be canonical.

The canonical statistics of this exponential family are $X_1$ and $X_2$
and the corresponding canonical parameters are $\psi_i = \log(\mu_i)$.
Linear functions of canonical parameters are again canonical so we can test
or produce fuzzy confidence intervals
for $\psi_1 - \psi_2 = \log(\mu_1 / \mu_2)$.

Introduce new parameters
\begin{align*}
   \psi_1 & = \eta + \theta
   \\
   \psi_2 & = \eta
\end{align*}
Then
$$
   X_1 \psi_1 + X_2 \psi_2
   =
   X_1 \theta + (X_1 + X_2) \eta
   =
   T(\mathbf{X}) \theta + U(\mathbf{X}) \eta
$$
Where
\begin{equation} \label{eq:t-and-u}
\begin{split}
   T(\mathbf{X}) & = X_1
   \\
   U(\mathbf{X}) & = X_1 + X_2
\end{split}
\end{equation}

It is a standard result that the conditional distribution of
$T(\mathbf{X})$ given $U(\mathbf{X})$ is
$$
   X_1 \mid X_1 + X_2
   \sim
   \BinomialDis\left(X_1 + X_2, \frac{\mu_1}{\mu_1 + \mu_2}\right)
$$
So the theory says we do the UMP or UMPU test based on this distribution
with $\mu_1 / (\mu_1 + \mu_2)$ as the parameter of interest
(Lehmann, TSH, 1st ed., pp.~140--142, gives further details).

\subsection{Two Independent Binomial Proportions} \label{sec:two-binom-indep}

Let $X_i \sim \BinomialDis(n_i, p_i)$ with $0 < p_i < 1$, for $i = 1$, $2$
be independent random variables.  We wish to compare the proportions $p_1$ and
$p_2$.  We cannot just test or produce fuzzy confidence intervals for
a function pulled out of the air, such as $p_1 - p_2$.  The parameter
we test must be canonical.

The canonical statistics of this exponential family are $X_1$ and $X_2$
and the corresponding canonical parameters are $\psi_i = \logit(p_i)$.
Linear functions of canonical parameters are again canonical so we can test
or produce fuzzy confidence intervals for $\psi_1 - \psi_2$.

As in the Poisson case we see that we can base the test on the conditional
distribution of $T(\mathbf{X})$ given $U(\mathbf{X})$, where these variables
are defined by \eqref{eq:t-and-u}.
This distribution is known although is is not ``nice'' and is
not a ``brand name'' distribution.
It is (Lehmann, TSH, 1st ed., pp.~142--143) the exponential family generated
by the hypergeometric distribution.
\begin{equation} \label{eq:hypergeo}
   \pr\{ X_1 = t \mid X_1 + X_2 = u \}
   =
   \frac{1}{c(\rho)} \binom{n_1}{t} \binom{n_2}{u - t} \rho^t,
   \qquad t = 0, \ldots, u,
\end{equation}
where
$$
   \rho = \frac{p_1 (1 - p_2)}{(1 - p_1) p_2}
$$
and
$$
   c(\rho)
   =
   \sum_{t = 0}^u \binom{n_1}{t} \binom{n_2}{u - t} \rho^t.
$$
So the theory says we do the UMP or UMPU test based on this distribution
with $\rho$ as the parameter of interest

This distribution is very hairy.  For starters the actual range of the
random variable $X_1 \mid X_1 + X_2$ is not 0 to $X_1 + X_2$ as
\eqref{eq:hypergeo} seems to indicate.  The binomial coefficients can
evaluate to zero.

We will leave the complexity of this model and move on to other models.
In principle it is just a one-parameter exponential family and hence ``nice''
in certain respects.  In practice, there is no readily available software
to calculate distribution and density functions, so this model requires
more effort in computer implementation.

\subsection{Two Independent Negative Binomial Variables}

Let $X_i \sim \NegativeBinomialDis(r_i, p_i)$ with $0 < r_i$ and
$0 < p_i < 1$, for $i = 1$, $2$ be independent random variables.
As in Section~\ref{sec:negbin} we are using the convention that the
sample space starts at zero.  We wish to compare the proportions $p_1$ and
$p_2$.  We cannot just test or produce fuzzy confidence intervals for
a function pulled out of the air, such as $p_1 - p_2$.  The parameter
we test must be canonical.

The canonical statistics of this exponential family are $X_1$ and $X_2$
and the corresponding canonical parameters are $\psi_i = \log(1 - p_i)$.
Linear functions of canonical parameters are again canonical so we can test
or produce fuzzy confidence intervals for $\psi_1 - \psi_2$.

As in the Poisson case we see that we can base the test on the conditional
distribution of $T(\mathbf{X})$ given $U(\mathbf{X})$, where these variables
are defined by \eqref{eq:t-and-u}.
This distribution is not fully explained in Lehmann, although
the $r_1 = r_2 = 1$ case is the subject of a homework problem in the
second edition.

Let's see what happens.  The joint distribution of the $X$'s is
\begin{align}
   f(x_1, x_2)
   & =
   \prod_{i = 1}^2
   \frac{\Gamma(x_i + r_i)}{\Gamma(r_i) x_i!} p_i^{r_i} (1 - p_i)^{x_i}
   \nonumber
   \\
   & =
   \exp(x_1 \psi_1 + x_2 \psi_2)
   \prod_{i = 1}^2
   \frac{\Gamma(x_i + r_i)}{\Gamma(r_i) x_i!} p_i^{r_i}
   \nonumber
   \\
   & =
   \exp(t \theta + u \eta)
   \frac{\Gamma(t + r_1)}{\Gamma(r_1) t!}
   \frac{\Gamma(u - t + r_2)}{\Gamma(r_2) (u - t)!}
   p_1^{r_1} p_2^{r_2}
   \label{eq:unnormal}
\end{align}
where
\begin{align*}
   t & = x_1
   \\
   u & = x_1 + x_2
   \\
   \theta & = \psi_1 - \psi_2
   \\
   \eta & = \psi_2
\end{align*}
It matters not that we have not specified $p_1$ and $p_2$ as a function
of $\theta$ and $\eta$.  We want to consider the conditional distribution
anyway.  Thought of as a function of $t$ for fixed $u$ and dropping all terms
that do not contain both parameters and $t$ we get
\begin{equation} \label{eq:neg-hyper-expo}
   f_\theta(t \mid u)
   =
   \frac{1}{c(\theta)}
   \exp(t \theta)
   \frac{\Gamma(t + r_1) \Gamma(u - t + r_2)}{t! (u - t)!},
   \qquad t = 0, \ldots, u,
\end{equation}
where
$$
   c(\theta)
   =
   \sum_{t = 0}^u
   \exp(t \theta)
   \frac{\Gamma(t + r_1) \Gamma(u - t + r_2)}{t! (u - t)!}.
$$
Hmmmm.  Also not a brand name family and not the exponential family generated
by the hypergeometric distribution we saw in the case of two independent
binomial proportions.  Still just another one-parameter exponential family.
No big deal.

Actually, Mathematica says this has some sort of relation to hypergeometric
functions, for example
\begin{verbatim}
In[4]:= Sum[ rho^t Gamma[t + r] Gamma[u - t + s] /
        (Gamma[t + 1] Gamma[u - t + 1]), {t, 0, u} ]

        Gamma[r] Gamma[s + u] Hypergeometric2F1[r, -u, 1 - s - u, rho]
Out[4]= --------------------------------------------------------------
                                 Gamma[1 + u]
\end{verbatim}
So Mathematica knows how to calculate the normalizing function and it
involves a ``hypergeometric $\vphantom{F}_2 F_1$ function.  Maybe we
should call this the exponential family generated by the negative
hypergeometric distribution.

According to
\begin{verbatim}
http://planetmath.org/encyclopedia/NegativeHypergeometricDistribution.html
\end{verbatim}
there is a \emph{negative hypergeometric distribution} having density
$$
   f(x) =
   \frac{ \binom{x+b-1}{x} \binom{W+B-b-x}{W-x} }{ \binom{W+B}{W} },
   \qquad x = 0, 1, \ldots, W,
$$
where $W$, $B$, and $b$ are positive integers.  The web page explains
that this is the distribution of the number of ``special items'' $X$
(from $W$ special items in the population) present before the $b$th
object from a population with $B$ items altogether.

Comparing with \eqref{eq:neg-hyper-expo} we see that the base measure
of the family does indeed have this form when $r_1$ and $r_2$ are integers.

\subsection{Testing Independence in a Two-by-two Table} \label{sec:exact}

This is the UMP/UMPU competitor for Fisher's exact test.  The data consist
of a matrix $X_{i j}$, $i = 1$, 2, $j = 1$, 2, that has a multinomial
distribution with sample size $n$ and cell probability matrix $p_{i j}$,
$i = 1$, 2, $j = 1$, 2.  This is also called a two-by-two contingency table.

The canonical statistics are the $X_{i j}$, but the canonical parameters are
not uniquely defined in terms of the $p_{i j}$ because the model is really
only three dimensional, not four, because the $X_{i j}$ sum to $n$.

As is well known, this is a three-dimensional exponential family, the
canonical statistics being any three of the four $X_{i j}$, the fourth
being determined from the other three by the requirement that the $X_{i j}$
sum to $n$.

In this problem (Lehmann, TSH, 1st ed., pp.~143--146) the marginals
are the statistics for the nuisance parameters,
and we can consider any other statistic
linearly independent of the marginals and the sum of all cells as the
statistic of interest.  Lehmann chooses
\begin{equation} \label{eq:exact}
\begin{split}
   T(\mathbf{X}) & = X_{1 1}
   \\
   U_1(\mathbf{X}) & = X_{1 1} + X_{1 2}
   \\
   U_2(\mathbf{X}) & = X_{1 1} + X_{2 1}
\end{split}
\end{equation}

The conditional distribution of $T(\mathbf{X})$ given the marginals
$U_1(\mathbf{X})$ and $U_2(\mathbf{X})$ is well known.  It is the
hypergeometric distribution involved in Fisher's exact test under the
null hypothesis of independence and under general null hypotheses is
the exponential family generated by the hypergeometric we encountered
in Section~\ref{sec:two-binom-indep}.
\begin{multline} \label{eq:hyper-fish}
   Pr\{X_{1 1} = t \mid X_{1 1} + X_{1 2} = u_1, X_{1 1} + X_{2 1} = u_2\}
   \\
   =
   \frac{1}{c(\rho)} \binom{u_1}{t} \binom{n}{u_2 - t} \rho^t,
   \qquad t = 0, \ldots, n
\end{multline}
where
$$
   \rho = \frac{p_{1 1} p_{2 2}}{p_{1 2} p_{2 1}}
$$
and
$$
   c(\rho) = \sum_{t = 0}^n \binom{u_1}{t} \binom{n}{u_2 - t} \rho^t.
$$
As in Section~\ref{sec:two-binom-indep} we note that this distribution
given by \eqref{eq:hyper-fish} often does not have the full range 0 to $n$
because the binomial coefficients may be zero.

\subsection{Different Answers to the Same Question in a Poll}

This section and the next give the UMP/UMPU/fuzzy competitors to the
analysis of correlated binomial proportions in Wild and Seber
(pp.~343--350).  The first considers different answers to the same
question on a poll.  This is a multinomial problem.  Say the categories
of interest have counts $X_1$ and $X_2$, then we know
$$
   X_1 \mid X_1 + X_2
   \sim
   \BinomialDis\left(X_1 + X_2, \frac{p_1}{p_1 + p_2}\right)
$$
and so the UMP/UMPU/fuzzy procedures are based on this distribution.

\subsection{Answers to Different Questions in a Poll}

Here again, like in Section~\ref{sec:exact}, we have a two-by-two table
with data $X_{i j}$ and cell probabilities $p_{i j}$ but the question of
interest is different.  Now we are interested in just the opposite question,
whether the marginals differ.  This in a sense (a rather vague sense)
interchanges the role of interest and nuisance parameters, what were interest
parameters in Section~\ref{sec:exact} are now nuisance parameters and vice
versa.

Ordinarily, this would be nonsense.  There is exactly one interest parameter.
The rest (in this case two) must be nuisance parameters.  So,
strictly speaking, they cannot be interchanged.  But a two-by-two table has
a redundant canonical statistic: there are four $X_{i j}$ but they sum to $n$
so only three are linearly independent.  So if we add the redundant statistic
to the statistics corresponding to parameters of interest we two sets of two
that can be interchanged.

It is clear that \eqref{eq:exact} could have been written with subscripts
1 and 2 interchanged, which would make $X_{2 2}$ the statistic of interest.
This tells us that here we should condition on $X_{1 1}$ and $X_{2 2}$
leaving either $X_{1 2}$ or $X_{2 1}$ as the statistic of interest.
Thus in this case the UMP/UMPU/fuzzy procedure is based on the distribution
$$
   X_{1 2} \mid X_{1 1}, X_{2 2}
   \sim
   \BinomialDis\left(n - X_{1 1} - X_{2 2}, \frac{p_{1 2}}{p_{1 2} + p_{2 1}}
   \right)
$$

And in hindsight we see that we have invented the conditional, exact,
uniformly most whatever (UMW) competitor of McNemar's test (Lindgren,
\emph{Statistical Theory}, pp.~381--383).

\section{Algorithms}

\subsection{Version 0.1}

After a great struggle, a very simple algorithm was decided on for
calculating $\phi(x, \alpha, \theta)$.  Given $\alpha$ and $\theta$,
calculate the appropriate $c_1$, $c_2$, $\gamma_1$, and $\gamma_2$
as follows.

First handle the special cases where $\alpha$ is zero or one
and $\theta$ is on the boundary of the parameter space, using
Theorem~\ref{th:umpu-end} above and the obvious fact that $\phi$ is
identically equal to one when $\alpha = 1$ and identically equal to zero
when $\alpha = 0$.

When in the general case $0 < \alpha < 1$ and $\theta$ not on the boundary
choose some $c_1$ and $c_2$ such that $c_1 \le E_\theta\{T(X)\} \le c_2$.
We pick the $c_1$ and $c_2$ that give, with randomization, an equal tailed
test, in the hope that this is close to correct.

Then we go into an infinite loop that does the following.
\begin{itemize}
\item Calculate $\gamma_1$ and $\gamma_2$ using the current guesses
for $c_1$ and $c_2$ and equations \eqref{eq:gamma-1} and \eqref{eq:gamma-2}
above.  If the results satisfy 
$0 \le \gamma_1 \le 1$ and $0 \le \gamma_2 \le 1$, then we are done
and stop the loop.
\item Otherwise, we change $c_1$ or $c_2$, the one corresponding to the
$\gamma_i$ that violates the constraints worst.  If this $\gamma_i$ is
negative, we move the $c_i$ out by one
(i.~e.\ decrease $c_1$ or increase $c_2$)
and if this $\gamma_i$ is greater than one, we move the $c_i$ in by one.
\end{itemize}
Actually we don't do an infinite loop, because we have no theorem that
says this algorithm converges, so we have a maximum iteration count
(default 10) and just give up when it is reached.  In the examples
we have done, there has been no need to increase the iteration count.

See \path{ump/src/umpubinom.c} for an example of this algorithm.
See \path{ump/tests/umpub.R} for the tests it passed.

\subsection{Version 0.3}

An attempt to implement the density of abstract randomized $P$-values
and test the implementation shows that
equations \eqref{eq:gamma-1} and \eqref{eq:gamma-2} are no good.
The exhibit catastrophic cancellation for small alpha.

So we back up to \eqref{eq:pi}, \eqref{eq:p12}, and \eqref{eq:m12}
and keep \eqref{eq:pi} but ignore the other two, replacing them with
\begin{subequations}
\begin{align}
   P_1 & = \pr_\theta\{ T(X) < C_1 \}
   \label{eq:P1}
   \\
   P_2 & = \pr_\theta\{ T(X) > C_2 \}
   \label{eq:P2}
   \\
   M_1 & = E_\theta\{ T(X) I_{(- \infty, C_1)}[T(X)] \}
   \label{eq:M1}
   \\
   M_2 & = E_\theta\{ T(X) I_{(C_2, \infty)}[T(X)] \}
   \label{eq:M2}
\end{align}
\end{subequations}
Then we replace \eqref{eq:umpu-a-general} and \eqref{eq:umpu-b-general} by
\begin{subequations}
\begin{align}
   P_1 + \gamma_1 p_1 + \gamma_2 p_2 + P_2
   & = \alpha
   \label{eq:umpu-a-general-too}
   \\
   M_1 +  \gamma_1 C_1 p_1 + \gamma_2 C_2 p_2 + M_2
   & = \alpha \mu
   \label{eq:umpu-b-general-too}
\end{align}
\end{subequations}
which solved for $\gamma_1$ and $\gamma_2$ give
\begin{subequations}
\begin{align}
   \gamma_1
   & =
   \frac{\alpha (C_2 - \mu) + (M_1 - C_2 P_1) + (M_2 - C_2 P_2)}
   {p_1 (C_2 - C_1)}
   \label{eq:gamma-1-too}
   \\
   \gamma_2
   & =
   \frac{\alpha (\mu - C_1) - (M_1 - C_1 P_1) - (M_2 - C_1 P_2)}
   {p_2 (C_2 - C_1)}
   \label{eq:gamma-2-too}
\end{align}
\end{subequations}
%
% Solve[ { P1 + gamma1 p1 + gamma2 p2 + P2 == alpha,
%     M1 + gamma1 C1 p1 + gamma2 C2 p2 + M2 == mu alpha }, { gamma1, gamma2 } ]
%
% gamma1 = (alpha (C2 - mu) + (M1 - C2 P1) + (M2 - C2 P2)) / (p1 (C2 - C1))
% gamma2 = (alpha (mu - C1) - (M1 - C1 P1) - (M2 - C1 P2)) / (p2 (C2 - C1))
%
% P1 + gamma1 p1 + gamma2 p2 + P2
% M1 + gamma1 C1 p1 + gamma2 C2 p2 + M2
%
% Solve[ 1 == (alpha (C2 - mu) + (M1 - C2 P1) + (M2 - C2 P2)) / (p1 (C2 - C1)),
%     alpha ]
% Solve[ 1 == (alpha (mu - C1) - (M1 - C1 P1) - (M2 - C1 P2)) / (p2 (C2 - C1)),
%     alpha ]
% Solve[ 0 == (alpha (C2 - mu) + (M1 - C2 P1) + (M2 - C2 P2)) / (p1 (C2 - C1)),
%     alpha ]
% Solve[ 0 == (alpha (mu - C1) - (M1 - C1 P1) - (M2 - C1 P2)) / (p2 (C2 - C1)),
%     alpha ]
%

This seems to work better, but honesty compels us to admit that this formula
also is subject to catastrophic cancellation.  Perusal of the source code
reveals several ad hoc bits of code that deal with special cases
in which the code without the adhockery fails due to catastrophic cancellation
or other problems with the inexactitude of floating point arithmetic.

It is fair to say that our code is far from an elegant and provably correct
solution to this problem.  We think the algorithm presented
on page~\pageref{pg:algo} would actually be better than the one we used
in all respects except that it takes time proportional to the sample size,
which was deemed unacceptable (perhaps wrongly).





\bibliographystyle{apalike}

\begin{thebibliography}{}

\bibitem[Lehmann, 1959]{lehmann}
\textsc{Lehmann, E.~L.} (1959).
\newblock \emph{Testing Statistical Hypotheses}.
\newblock Wiley, New York.
\newblock (2nd ed., Wiley, 1986; Springer, 1997).

\end{thebibliography}

\end{document}




\section{Computing}

\subsection{Nice Models}

In this section we take another shot at ``nice models'' by which we mean
those for which R has functions for calculating most things about the
distribution.  Of the models mentioned above, the nice models are the
binomial, Poisson, and negative binomial.  The exponential families
generated by the hypergeometric and negative hypergeometric distributions
are not ``nice.''

Let $F_\theta$ and $F_\theta^{-1}$ denote the cumulative distribution
and quantile functions as implemented in R (for example, \texttt{pbinom}
and \texttt{qbinom})
$$
   F_\theta^{-1}(p) = \inf \set{ x : F_\theta(x) \ge p }
$$
Define also an analogous ``$M$'' function for the mean
$$
   M_\theta(x) = \sum_{y = 0}^x y f_\theta(y)
$$
and its inverse
$$
   M_\theta^{-1}(s) = \inf \set{ x : M_\theta(x) \ge s }
$$
As discussed in the sections on the models, these ``$M$'' functions are
implemented in R as follows
\begin{verbatim}
mbinom <- function(x, n, p)
    n * p * pbinom(x - 1, n - 1, p)
mpois <- function(x, mu)
    mu * ppois(x - 1, mu)
mnbinom <- function(x, r, p)
    (1 - p) / p * pnbinom(x - 1, r + 1, p)
\end{verbatim}
The corresponding inverse functions are
\begin{verbatim}
nbinom <- function(s, n, p)
    1 + qbinom(s / (n * p), n - 1, p)
npois <- function(s, mu)
    1 + qpois(s / mu, mu)
nnbinom <- function(s, r, p)
    1 + qnbinom(s * p / (1 - p), r + 1, p)
\end{verbatim}

Then we can rewrite write the key
equations \eqref{eq:gamma-1} and \eqref{eq:gamma-2} as
\begin{align*}
   1 - \gamma_1
   & =
   \frac{(1 - \alpha) (C_2 - \mu) + m_{1 2} - C_2 p_{1 2}}
   {f_\theta(C_1) (C_2 - C_1)}
   \\
   1 - \gamma_2
   & =
   \frac{(1 - \alpha) (\mu - C_1) - m_{1 2} + C_1 p_{1 2}}
   {f_\theta(C_2) (C_2 - C_1)}
\end{align*}
where
\begin{align*}
   p_{1 2} & = F_\theta(C_2 - 1) - F_\theta(C_1)
   \\
   m_{1 2} & = M_\theta(C_2 - 1) - M_\theta(C_1)
\end{align*}
In order to have the gammas in range we need to satisfy the inequalities
\begin{align*}
   0
   & \le
   \frac{(1 - \alpha) (C_2 - \mu) + m_{1 2} - C_2 p_{1 2}}
   {f_\theta(C_1) (C_2 - C_1)}
   \le
   1
   \\
   0
   & \le
   \frac{(1 - \alpha) (\mu - C_1) - m_{1 2} + C_1 p_{1 2}}
   {f_\theta(C_2) (C_2 - C_1)}
   \le
   1
\end{align*}
or
\begin{gather*}
   0
   \le
   (1 - \alpha) (C_2 - \mu) + m_{1 2} - C_2 p_{1 2}
   \le
   f_\theta(C_1) (C_2 - C_1)
   \\
   0
   \le
   (1 - \alpha) (\mu - C_1) - m_{1 2} + C_1 p_{1 2}
   \le
   f_\theta(C_2) (C_2 - C_1)
\end{gather*}
Adding these inequalities to eliminate $m_{1 2}$ gives
$$
   0 \le (1 - \alpha - p_{1 2}) (C_2 - C_1)
   \le \bigl[ f_\theta(C_1) + f_\theta(C_2) \bigr] (C_2 - C_1)
$$
or
$$
   0 \le 1 - \alpha - F_\theta(C_2 - 1) + F_\theta(C_1)
   \le f_\theta(C_1) + f_\theta(C_2)
$$
or
\pagebreak[3]
\begin{subequations}
\begin{equation} \label{eq:fred-a}
   F_\theta(C_2 - 1) - F_\theta(C_1)
   \le
   1 - \alpha
   \le
   F_\theta(C_2) - F_\theta(C_1 - 1)
\end{equation}
Similarly, eliminating $p_{1 2}$ gives
\begin{equation} \label{eq:fred-b}
   M_\theta(C_2 - 1) - M_\theta(C_1)
   \le
   (1 - \alpha) \mu
   \le
   M_\theta(C_2) - M_\theta(C_1 - 1)
\end{equation}
\end{subequations}
Note that in hindsight \eqref{eq:fred-a} and \eqref{eq:fred-b} are obvious.

In order to satisfy \eqref{eq:fred-a} and \eqref{eq:fred-b} we must have
$$
   F_\theta^{-1}\bigl\{ F_\theta(C_1 - 1) + 1 - \alpha \bigr\}
   \le
   C_2
   \le
   1 + F_\theta^{-1}\bigl\{ F_\theta(C_1) + 1 - \alpha + \epsilon \bigr\},
   \qquad \forall \epsilon > 0
$$
and
$$
   M_\theta^{-1}\bigl\{ M_\theta(C_1 - 1) + (1 - \alpha) \mu \bigr\}
   \le
   C_2
   \le
   1 + M_\theta^{-1}\bigl\{ M_\theta(C_1) + (1 - \alpha) \mu + \epsilon \bigr\},
   \qquad \forall \epsilon > 0
$$
the $\forall \epsilon > 0$ being an unavoidable hassle caused by the
interaction of discreteness and the right continuity convention.
It's too bad the R ``q'' functions don't come with a flag
for either right or left continuity.
Basically, these are just \eqref{eq:fred-a} and \eqref{eq:fred-b} ``inverted''
which is what we need to use in the computer, but we will stick with the
simpler \eqref{eq:fred-a} and \eqref{eq:fred-b} for analysis.

Now suppose we are interested in one fixed $x$ and $\theta$.  Suppose,
for concreteness that $x < \mu$ so if $\phi(x, \alpha, \theta)$ is
strictly between zero and one we must have $x = C_1$.  Then our inequalities
can be used to determine the range of $\alpha$ and $C_2$ for which we
get solutions.

For starters, we must have $\mu < C_2 \le x_{\text{max}}$, where 
$x_{\text{max}}$ is the upper bound of the support (perhaps $\infty$).
And this implies from \eqref{eq:fred-a} that
$$
   F_\theta(C_1 - 1) \le \alpha \le 1 - F_\theta(\mu) + F_\theta(C_1)
$$
and from \eqref{eq:fred-b} that
$$
   \frac{M_\theta(C_1 - 1)}{\mu}
   \le
   \alpha
   \le
   1 - \frac{M_\theta(\mu) - M_\theta(C_1)}{\mu}
$$

\section{New Computing}

In this section (November, 2002) we look at the analog of following
the curve $\theta \mapsto \phi(x, \alpha, \theta)$ to get
the membership function of a fuzzy confidence interval, the same
way we followed $\alpha \mapsto \phi(x, \alpha, \theta)$ to get the
cumulative distribution function of a fuzzy $P$-value.
The latter is piecewise linear.  The former isn't.  But how hard is it?

We know how to get started from the ``endpoint'' analysis
(Section~\ref{sec:endpoint}).  The membership function of the fuzzy
confidence interval goes to $1 - \alpha$ or to 0 as $\theta$ goes to
its upper or lower limit, depending on the value of $x$.
(This means we are still in ``nice'' models, one-parameter exponential
families with canonical statistic taking values on a range of consecutive
integers).

So we can start at the edge of the parameter space and go from there.
What do equations \eqref{eq:gamma-1} and \eqref{eq:gamma-2} do
\emph{considered as functions of $\theta$ for fixed $\alpha$}?

Let's start with the case of $\theta$ near the boundary.  Suppose
the sample space of the canonical statistic is $\{ 0, 1, \ldots \}$
(so we are in the situation
analyzed in Section~\ref{sec:endpoint}, but we want to know more than
just $o(\mu)$ used there.  For small enough $\theta$ we do have $C_1 = 0$
and $C_2 = 1$ so $p_{1 2} = m_{1 2} = 0$ and
equations \eqref{eq:gamma-1} and \eqref{eq:gamma-2} become
\begin{subequations}
\begin{align}
   1 - \gamma_1
   & =
   \frac{(1 - \alpha) (1 - E_\theta\{T(X)\})}{\pr_\theta\{T(X) = 0\}}
   \label{eq:gamma-1-mbogo}
   \\
   1 - \gamma_2
   & =
   \frac{(1 - \alpha) E_\theta\{T(X)\}}{\pr_\theta\{T(X) = 1\}}
   \label{eq:gamma-2-mbogo}
\end{align}
\end{subequations}

For the $\text{Binomial}(n, p)$ distribution these become
\begin{subequations}
\begin{align}
   1 - \gamma_1
   & =
   \frac{(1 - \alpha) (1 - n p)}{(1 - p)^n}
   \label{eq:gamma-1-mbogo-bino}
   \\
   1 - \gamma_2
   & =
   \frac{(1 - \alpha) n p}{n p (1 - p)^{n - 1}}
   \nonumber
   \\
   & =
   \frac{(1 - \alpha)}{(1 - p)^{n - 1}}
   \label{eq:gamma-2-mbogo-bino}
\end{align}
\end{subequations}
So we see these equations, though nonlinear, are not horribly intractable.
They are good for the $\theta$ for which their values stay between zero
and one.  For \eqref{eq:gamma-2-mbogo-bino} this is for
$$
   0 \le p \le 1 - (1 - \alpha)^{1 / (n - 1)}
$$
at the upper end of this range where $\gamma_2 = 1$, we can make the
change $C_2 = 2$ and $\gamma_2 = 0$ without affecting the critical function.

With either of the $C$'s changed we get the more general equations
\eqref{eq:gamma-1} and \eqref{eq:gamma-2} now in the form
\begin{subequations}
\begin{align}
   1 - \gamma_1
   & =
   \frac{(1 - \alpha) [C_2 - \mu(\theta)] - E_\theta\{I_{1 2}(X) [C_2 - X]\}}
   {f_\theta(C_1) (C_2 - C_1)}
   \label{eq:gamma-1-muggs}
   \\
   1 - \gamma_2
   & =
   \frac{(1 - \alpha) [\mu(\theta) - C_1] - E_\theta\{I_{1 2}(X) [X - C_1]\}}
   {f_\theta(C_2) (C_2 - C_1)}
   \label{eq:gamma-2-muggs}
\end{align}
\end{subequations}
where
$$
   I_{1 2}(x) = \begin{cases} 1, & C_1 < x < C_2 \\
   0, & \text{otherwise} \end{cases}
$$
It follows from the general properties of exponential families that
if $\theta$ is the canonical parameter
$$
   \frac{\partial}{\partial \theta} f_\theta(x) = [x - \mu(\theta)] f_\theta(x)
$$
$$
   \frac{\partial}{\partial \theta} \mu(\theta) = \sigma^2(\theta)
$$
Hence
\begin{align*}
   \frac{\partial}{\partial \theta} (1 - \gamma_1)
   & =
   \frac{1}{f_\theta(C_1) (C_2 - C_1)} \times \biggl(
   - (1 - \alpha) \sigma^2(\theta)
   \\
   & \qquad
   - E_\theta\{I_{1 2}(X) [C_2 - X] [X - \mu(\theta)]\}
   \\
   & \qquad
   + (1 - \alpha) [C_2 - \mu(\theta)] [\mu(\theta) - C_1]
   \\
   & \qquad
   - E_\theta\{I_{1 2}(X) [C_2 - X] [\mu(\theta) - C_1]\} \biggr)
   \\
   & =
   \frac{1}{f_\theta(C_1) (C_2 - C_1)} \times \biggl(
\end{align*}




In considering the distribution of the fuzzy $P$-value, we look at
$\phi(x, \alpha, \theta)$ as a function of $\alpha$ for fixed $x$ and $\theta$,
hence at \eqref{eq:gamma-ump} in the same way.  Now $T(x)$ will be a
$(1 - \alpha)$-th quantile if
$$
   \pr_\theta\{ T(X) > T(x) \} \le \alpha \le 
   \pr_\theta\{ T(X) \ge T(x) \}.
$$
Hence
\begin{align*}
   \phi(x, \alpha, \theta)
   & =
   \begin{cases}
   0, & \pr_\theta\{ T(X) \ge T(x) \} \ge \alpha
   \\
   1, & \pr_\theta\{ T(X) > T(x) \} \le \alpha
   \\
   \frac{\alpha - \pr_\theta\{T(X) > T(x)\}}{\pr_\theta\{T(X) = T(x)\}}, &
   \text{otherwise}
   \end{cases}
\end{align*}
Since this is linear where not zero or one,
it is the distribution function of a uniform random variable
\begin{equation} \label{eq:fpv-ump-dist}
   P \sim \text{Uniform}\bigl( \pr_\theta\{ T(X) > T(x) \},
   \pr_\theta\{ T(X) \ge T(x) \} \bigr),
\end{equation}
that is, the fuzzy $P$-value is uniformly distributed on the interval
of values that you figure it should spread it over.  Intuitively obvious, but
we had to check to be sure.  Please don't feel put upon by our belaboring
the obvious, because the UMPU two-tailed case is not quite so obvious, and
understanding the UMP one-tailed case helps.

